\documentclass[12pt,a4paper]{report}
\usepackage[margin=20mm]{geometry}
\usepackage{fancyhdr}
\usepackage{lastpage}
\usepackage{parskip}
\usepackage{titlesec}
\usepackage{color}

\usepackage[T1]{fontenc}
\usepackage[utf8]{inputenc}
\usepackage[scale=0.9]{sourcecodepro}

\definecolor{gray75}{gray}{0.5}
\newcommand{\hsp}{\hspace{20pt}}

\titleformat{\chapter}[hang]{\Huge\bfseries}{\textcolor{gray75}{Chapter \thechapter}\hsp \textcolor{gray75}{---}\hsp}{0pt}{\sc \bfseries}
%\titlespacing\chapter{0pt}{-4em}{0pt}
\titlespacing*{\chapter}{0pt}{-30pt}{0pt}

\titleformat{\section}
{ \large  \bfseries}{\thesection}{0.5em}{}
\titlespacing\section{0pt}{12pt plus 4pt minus 2pt}{0pt plus 2pt minus 2pt}

\renewcommand\labelitemi{--}



\setlength{\parskip}{0.4em}
\begin{document}

\chapter*{Progress Report}

\begingroup
\renewcommand*{\arraystretch}{1.5}
\begin{tabular}{@{} l l @{}}
  \textbf{Name:}                & Phoebe Nichols (pmn29) \\
  \textbf{College:}             & Churchill College \\
  \textbf{Project Title:}       & An Implementation of Prolog \\
  \textbf{Project Supervisor:}  & Prof. Alan Mycroft (am21)\\
  \textbf{Director of Studies:} & Dr J.~K. Fawcett (jkf21) \\
  \textbf{Project Overseers}:   & Prof. Andrew Pitts (amp12) \& Dr Rafal Mantiuk (rkm38) \\
\end{tabular}
\endgroup
% \vspace{1em}

\section*{Project summary}
My project is about implementing Prolog using an abstract machine. The abstract machine should execute Prolog using an instruction set that is chosen as part of the project. The success criterion of the project is about writing a correct implementation rather than a highly performant one, although performance would be a nice bonus.

My success criterion is to implement the following components for a basic subset of Prolog:
\begin{itemize}
\item Lexer
\item Parser
\item Translator from parse tree to byte-code
\item Abstract machine to execute this byte-code
\end{itemize}
My extensions are to add various optimisations to the compiler. The optimisations suggested in my proposal are last call optimisation, determinacy analysis, mode analysis, and type checking.

\section*{Completed work}

I have achieved my success criterion: I have a working implementation of all components. The lexer and parser were implemented using automated lexer and parser generations, and the translator and abstract machine were implemented in OCaml. My implementation passes tests such as backtracking depth-first search, quick-sort, and the N-queens problem.

The abstract machine uses an instruction set that I designed based on the most commonly used instruction set for Prolog, but makes some simplifying choices and also includes additional instructions to support features such as arithmetic and cut. The abstract machine looks to be reasonably but not exceptionally performant; it is currently 5 times slower than SWI-Prolog to solve the 12-queens problem.

I have also implemented some extension features. I have implemented last call optimization (this is a generalised form of tail recursion optimisation for logic programming languages). I am currently adding a type system: I have added syntax for the types to the language, but not yet finished the type checking.


\section*{Project schedule}

The project is currently on schedule. I have met all milestones to date, but am not ahead on any. 

\section*{Unexpected difficulties}

None.


\end{document}